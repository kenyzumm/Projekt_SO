\documentclass[a4paper]{article}
\usepackage[T1]{fontenc}
\usepackage[utf8]{inputenc}
\usepackage{lmodern}
\usepackage{tikz}
\usetikzlibrary{shapes.geometric, arrows.meta, positioning, fit, calc, decorations.pathmorphing}

\begin{document}

% --- STRONA 1: GRAF DZIAŁANIA ---
\begin{figure}[p]
\centering
\begin{tikzpicture}[
    >={Stealth[length=2mm]},
    font=\sffamily\small,
    proc/.style={rectangle, rounded corners=5pt, draw, fill=blue!10, minimum width=2.8cm, minimum height=1.3cm, align=center, thick},
    mainproc/.style={rectangle, rounded corners=5pt, draw, fill=green!10, minimum width=3.8cm, minimum height=1.6cm, align=center, thick},
    resource/.style={trapezium, trapezium left angle=70, trapezium right angle=110, draw, fill=gray!10, align=center, minimum width=2.5cm},
    pipe/.style={draw, fill=orange!10, cylinder, shape border rotate=0, minimum width=0.7cm, minimum height=1.2cm},
    sig_arrow/.style={->, color=red, thick, dashed},
    data_arrow/.style={->, thick},
    label_box/.style={font=\tiny\itshape, color=black}
]

% 1. PROCESY (Zgodnie z main.c)
\node[mainproc] (Main) at (0,0) {\textbf{Main (Rodzic)}\\parent\_send\_control()\\(Zarz\k{a}dzanie)};
\node[proc] (P3) at (5,-5) {\textbf{Proces 3}\\(Producent)\\p3\_notify\_handler};
\node[proc] (P2) at (0,-5) {\textbf{Proces 2}\\(Przetwarzanie)\\p2\_out\_signal\_handler};
\node[proc] (P1) at (-5,-5) {\textbf{Proces 1}\\(Konsument)\\p1\_notify\_handler};

% 2. ZASOBY IPC (DANE)
\node[resource] (SHM) at (2.5,-7.5) {Pami\k{e}\'{c} Wsp\'{o}ldzielona\\(P2 $\leftrightarrow$ P3)};
\node[resource] (MQ) at (-2.5,-7.5) {Kolejka Komunikat\'{o}w\\(P1 $\leftrightarrow$ P2)};

% 3. POTOKI (PIPES) DLA SYGNA\L \'{O}W
\node[pipe] (Pipe3) at (3,-2.5) {pipes[2]};
\node[pipe] (Pipe2) at (0,-2.5) {pipes[1]};
\node[pipe] (Pipe1) at (-3,-2.5) {pipes[0]};

% --- PRZEP\L YW DANYCH ---
\draw[data_arrow] (5,-9) node[below]{Plik / STDIN} -- (P3);
\draw[data_arrow] (P3) -- node[right, label_box]{Semafory (P/V)} (SHM);
\draw[data_arrow] (SHM) -- node[left, label_box]{shm->len} (P2);
\draw[data_arrow] (P2) -- node[right, label_box]{msgsnd()} (MQ);
\draw[data_arrow] (MQ) -- node[left, label_box]{msgrcv()} (P1);
\draw[data_arrow] (P1) -- (-5,-9) node[below]{STDOUT};

% --- \L A\~NCUCH SYGNA\L \'{O}W ---

% Krok 1: Przechwycenie sygna\l u przez P2
\node (ExtSig) at (0,-8) [draw, dashed, inner sep=2pt]{U\.zytkownik (Ctrl+Z)};
\draw[sig_arrow] (ExtSig) -- node[midway, right, font=\tiny]{1. SIGTSTP} (P2);

% Krok 2: P2 -> Main (wejście od dołu)
\draw[sig_arrow] (P2.west) .. controls (-3,-3) and (-2,-3) .. 
    node[left, pos=0.3, align=right, font=\tiny]{2. kill(ppid, SIGUSR2)\\Po\'{s}rednictwo} 
    (Main.south);

% Krok 3: Main -> Pipes
\draw[->, color=blue, thick] (Main) -- (Pipe1);
\draw[->, color=blue, thick] (Main) -- (Pipe2);
\draw[->, color=blue, thick] (Main) -- (Pipe3);
\node[label_box] at (0,-1.3) {3. write(command) do rur};

% Krok 4: Kaskada (Notify)
\draw[sig_arrow] (Main.east) .. controls (5.5,-1) and (5.5,-3) .. 
    node[right, font=\tiny]{4. notify(P3)} (P3.north);
\draw[sig_arrow] (P3) -- node[above, font=\tiny]{5. notify(P2)} (P2);
\draw[sig_arrow] (P2) -- node[above, font=\tiny]{6. notify(P1)} (P1);

% Odczyty z rur
\draw[dashed, ->, gray] (Pipe3) -- (P3);
\draw[dashed, ->, gray] (Pipe2) -- (P2);
\draw[dashed, ->, gray] (Pipe1) -- (P1);

\end{tikzpicture}
\caption{Schemat przep\l ywu sygna\l \'{o}w i danych zgodnie z implementacj\k{a}.}
\end{figure}

\newpage % --- STRONA 2: LEGENDA ---

\begin{figure}[t]
\centering
\vspace*{2cm}
\begin{tikzpicture}[font=\sffamily\small]
\node[draw, fill=white, inner sep=20pt, thick] {
    \begin{tabular}{ll}
        \textcolor{red}{$-\!-\!\to$} & \textbf{Sygna\l y steruj\k{a}ce} (SIGUSR1/2) - Mechanizm powiadamiania o zmianie stanu. \\
        \rule{0pt}{4ex} \textcolor{blue}{$\longrightarrow$} & \textbf{Zapis do potok\'{o}w} - Przesy\l anie konkretnej komendy (STOP/CONT) przez Main. \\
        \rule{0pt}{4ex} $\longrightarrow$ & \textbf{Przep\l yw danych} - Dane odczytane, licznik znak\'{o}w, wyj\'{s}cie. \\
        \rule{0pt}{4ex} \textcolor{orange}{Cylinder} & \textbf{Potoki (Pipes)} - Buforowanie komend dla ka\.zdego procesu. \\
        \rule{0pt}{4ex} \textcolor{gray}{Trapez} & \textbf{Zasoby IPC System V} - Pami\k{e}\'{c} wsp\'{o}ldzielona (SHM) i Kolejka komunikat\'{o}w (MQ). \\
        \rule{0pt}{4ex} \textbf{Notify} & \textbf{Kaskada} - Main budzi P3, P3 budzi P2, P2 budzi P1 (zgodnie z zadaniem). \\
    \end{tabular}
};
\end{tikzpicture}
\end{figure}

\end{document}
